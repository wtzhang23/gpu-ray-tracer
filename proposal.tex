\documentclass[11pt]{article}
\usepackage[margin=1in]{geometry}
\usepackage{graphicx}
\usepackage{subcaption}
\title{Project Proposal: GPU accelerated ray tracing}
\author{William Zhang}
\date{\today}

\begin{document}
\maketitle
\section{Objective}
By the end of this project, I should have a working ray tracer capable of generating scenes in real time.

\section{Requirements}
\begin{enumerate}
    \item GPU: I own one in my own machine which will allow me to see whether my product is working
\end{enumerate}

\section{Summary}
Ray tracing involves casting rays corresponding to each pixel on the screen into the
world and coloring the pixel dependent on the objects that it intersects. As each pixel
is independent of one another, ray tracing is highly parallelizable. However, doing so
on the GPU does pose problems due to secondary rays that are constructed during reflection
and refraction. In this project, I will learn how to deal with these challenges and create
a GPU-accelerated renderer.

\section{Timeline}
Below is my proposed timeline for this project. Italicized bullet points indicate topics pertaining
to my graphics final project
\begin{enumerate}
    \item \textbf{Wednesday, April 21:} Implement the ray tracer with support of
          \begin{enumerate}
              \item Displaying images on the GPU to the monitor
              \item Triangle mesh intersection
              \item Phong illumination model
          \end{enumerate}
    \item \textbf{Sunday, April 24:} Improve on the ray tracer
          \begin{enumerate}
              \item Secondary ray calculation
              \item Bounding volume heirarchy tree for pruning intersection search space
              \item \textit{Texture mapping of surfaces}
          \end{enumerate}
    \item \textbf{Wednesday, April 28:} Implement a scene
          \begin{enumerate}
              \item \textit{Procedural generation of a world}
              \item \textit{World, camera, and local coordinate spaces}
              \item \textit{GUI and user input and output}
          \end{enumerate}
    \item \textbf{Monday, May 1:} Real-time optimizations and presentation preparation
\end{enumerate}

\section{Measurements}
\begin{enumerate}
    \item The primary measurement I will be taking is the frame rate of the program. Should my
          program reach a framerate greater than or equal to 60, I would consider the program to be functional
          in real time. I anticipate that simple GPU acceleration would not suffice to reach this mark,
          and substantial optimizations involving pruning objects from intersection is needed.
\end{enumerate}

\end{document}